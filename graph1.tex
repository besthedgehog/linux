\documentclass{standalone}
\usepackage{tikz}
\usetikzlibrary{shapes,arrows,positioning}
\usepackage[utf8]{inputenc}
\usepackage[T1]{fontenc}
\usepackage[english,russian]{babel}

\begin{document}
\begin{tikzpicture}[
    node distance = 2cm, % Расстояние между узлами
    oval/.style={draw, ellipse}, % Стиль для овала
    square/.style={draw, minimum size=2cm}, % Стиль для квадрата
    arrow/.style={->, >=latex}, % Стиль для стрелок
]

% Узлы
\node[square] (center) {На основании чего система примет решение что пакет не подходит?};
\node[oval, below=of center] (dependencies) {\textcyrillic{Зависимости}};
\node[oval, left=of dependencies] (version) {Версия системы};
\node[oval, right=of dependencies] (architecture) {Архитектура процессора};
\node[oval, below=of dependencies, xshift=+3cm, yshift=1cm] (security) {Ограничения безопасности};
\node[oval, below left=of security, xshift=-0.5cm, yshift=+1cm] (conflicts) {Конфликты с другими пакетами};

% Стрелки
\draw[arrow] (center) -- (dependencies);
\draw[arrow] (center) -- (version);
\draw[arrow] (center) -- (architecture);
\draw[arrow] (center) -- (security);
\draw[arrow] (center) -- (conflicts);

\end{tikzpicture}
\end{document}
